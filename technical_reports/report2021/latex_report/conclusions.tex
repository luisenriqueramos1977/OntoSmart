\section{Discussion of this Report}

From the very beginning of this research, we have searched for an implementation approach, which means that we made efforts to validate by implementation the conclusions made in this Section. We now go through each of these, discussing our findings.

\noindent \textbf{\gls{owl} is not expressive enough for representing knowledge, rules and restrictions of the manufacturing domain:} 

We initiated our research with the assumption that \gls{owl} would be enough for our general purpose. Very early in the research process it became clear that, although \gls{owl} provides a suitable level of decidability, in our practical application of design validation, conclusions from certain kinds of constraints would not be possible. This happened in the given examples because \gls{owl} does not support ternary relations, limiting the expressiveness of this language. In a more complex scenario, which could include parthood relations (e.g. assemblies), and heterogeneous metric systems, \gls{owl} could therefore become useless. Moreover, it is necessary to highlight that \gls{owl} is frequently mentioned as the chosen language to develop manufacturing ontologies. But in most research we found that no statement on \gls{owl} limitations was provided. Therefore we can also conclude that this very important flaw has been largely neglected. According to our results this may happen because in most cases the proposed ontologies were only lightweight ontologies, lacking the axiom definitions required by heavyweight ontologies.  From our point of view, given that most researchers only developed ontologies at the lightweight level, the reasoning ability of \gls{owl} was not considered, consequently the flaw of this language was not evidenced in those studies. However, there are enough studies that refer to the limitation of \gls{owl} for knowledge representation, even in products and processes.

\noindent \textbf{Hyperontology, within \gls{ulo} and Modularity:} 

In most of the studies reviewed, we also realized a lack of proper methodologies for ontology development. Some authors highlight their proposed ontologies as \gls{ulo}s, and as we have stated in the previous Sections, this is a result of a misconception. Concerning modularity, or a modular framework, we considered the early work of \gls{tove} and \gls{swop}. However, most other research has made no mention of this method. Therefore we could state that they consider an individual ontology sufficient to reach their goals. In our case, both approaches were studied, intending to make use of existing ontologies. Then a methodology was proposed which included an algorithm that allowed us to generate a new type of module called hyper-modules. This group of modules constitutes what we have now called a Hyperontology. For us, this is an empirical result in the sense that we did not expect to find a consistent structure of modules or Hyperontology in the ontology network that we evaluated. However, this result was found on our proposal based on a clear algorithm that could be validated in relation to other sets of ontologies selected from a consistent domain. We would expect that similar results will be obtained in other domains.

\noindent \textbf{Heterogeneity, as a fact:} 

As mentioned in the first part of this Section, from the very beginning we were interested in making the most intensive use of the reasoning power of \gls{owl} within the Semantic Web framework. This intention led us to discover the flaws of this approach at an early stage as well. Therefore, we had to perform a deeper search into logics, languages, and related software tools. We found that \gls{hets} and its support of \gls{casl} provided an initial and basic support to extend the \gls{owl} language and its logics in relation to other logics, which would permit the achievement of more expressiveness. We cannot omit the fact that, when increasing expressiveness, we lose decidability, and so when working in cases related to manufacturing it is necessary to carefully evaluate each use in order to decide if the use case is restricted enough so that moving onto another logic can be avoided.  For instance, if we consider heterogeneous metric systems, it is most probable that we have a situation where we will need more expressive languages and logics than \gls{owl}. However, if we decide to use only one metric system, then we can avoid this scenario.

\noindent \textbf{Manufacturing processes are more than prismatic parts and shape fabrication:}  

Another point worth mentioning is that from our review of the literature, we found that most studies had centered their attention on prismatic parts and the shape fabrication procedure. However, manufacturing has other processes such as assembling, disassembling, forming, folding, and chemical procedures, among others, which to date have not been properly studied. Only one of the studies found mentioned assembly and disassembly \cite{vegetti_pronto:_2011}, whereas others did not make mention of this. Our research, as most other research mentioned here, was centered on prismatic parts, and its intention was not to demonstrate omissions on the part of other studies -- nevertheless, we considered it necessary to mention in these conclusions with the intention of identify open discussions for future studies.

\noindent \textbf{Methodological Integration is the key:}

Besides the limitations of \gls{owl} and its logics, \gls{owl} has considerable advantages, so it should not be left aside. On the contrary, we have integrated it into our proposed methodological framework. This methodological framework additionally includes methods for generating sets of hyper-modules constituting our hyper-ontology. This structure intends to use the knowledge inferred from one source ontology as the asserted knowledge of a target ontology. Furthermore, given that reasoning is a highly demanding task in Semantic Manufacturing, it is possible that \gls{owl} expressiveness may prove insufficient, and so a heterogeneous branch was included in our methodology in order to provide enough expressiveness to our framework when needed. The result of this integration was, first, a consistent Hyperontology, or network of modules, that linked most of the selected ontologies. Most of the concepts obtained by the proposed methodology were mentioned previously by other authors, moreover there were some other concepts that we also considered significant for the manufacturing domain.  Second, we also validated our methodology with the use case of providing a basic \gls{cad} design quality checker. 


\section{Opportunities for Future Research}

We consider that future research originating from our study can be performed in two ways: first, given that we have observed that the modeling pursued within the community of Semantic Manufacturing has omitted most of the manufacturing process, we recommend to move on to the production process of prismatic parts. This means that other processes, such as disaggregation, parts assembly, chemical, services, among others, should be included. On the other hand, from the Ontological Engineering point of view, the metrics and algorithm proposed by us for modules should be tested in other scenarios in order to determine if similar results can be obtained. Ontological Engineers should take special attention to the heterogeneity framework presented in this research, its \gls{casl} language and \gls{hets} tool, because the shortcoming of \gls{owl} can appear in other domains other than manufacturing. Ontologists should therefore be prepared to make use of this framework when necessary.

Finally, more complex products and processes than the examples provided in Sections \ref{section4.4} and \ref{section4.5} have to be studied. It is important to consider that from our Hyperontology depicted in Fig. \ref{figure4-28}, only one hyper module out of 12, and one ontology out of 8 were considered for this example. This represents less than 10 percent of the whole network. Thus, more complex scenarios need to be addressed in order to explore and evaluate the inclusion of more ontologies.